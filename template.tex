% Options for packages loaded elsewhere
% Options for packages loaded elsewhere
\PassOptionsToPackage{unicode}{hyperref}
\PassOptionsToPackage{hyphens}{url}
%
\documentclass[
  a4paper,
  oneside,
  open=any]{scrbook}
\usepackage{xcolor}
\usepackage[top=30mm,left=25mm,right=25mm,bottom=30mm]{geometry}
\usepackage{amsmath,amssymb}
\setcounter{secnumdepth}{-\maxdimen} % remove section numbering
\usepackage{iftex}
\ifPDFTeX
  \usepackage[T1]{fontenc}
  \usepackage[utf8]{inputenc}
  \usepackage{textcomp} % provide euro and other symbols
\else % if luatex or xetex
  \usepackage{unicode-math} % this also loads fontspec
  \defaultfontfeatures{Scale=MatchLowercase}
  \defaultfontfeatures[\rmfamily]{Ligatures=TeX,Scale=1}
\fi
\usepackage{lmodern}
\ifPDFTeX\else
  % xetex/luatex font selection
  \setmainfont[]{Public Sans}
\fi
% Use upquote if available, for straight quotes in verbatim environments
\IfFileExists{upquote.sty}{\usepackage{upquote}}{}
\IfFileExists{microtype.sty}{% use microtype if available
  \usepackage[]{microtype}
  \UseMicrotypeSet[protrusion]{basicmath} % disable protrusion for tt fonts
}{}
\makeatletter
\@ifundefined{KOMAClassName}{% if non-KOMA class
  \IfFileExists{parskip.sty}{%
    \usepackage{parskip}
  }{% else
    \setlength{\parindent}{0pt}
    \setlength{\parskip}{6pt plus 2pt minus 1pt}}
}{% if KOMA class
  \KOMAoptions{parskip=half}}
\makeatother
% Make \paragraph and \subparagraph free-standing
\makeatletter
\ifx\paragraph\undefined\else
  \let\oldparagraph\paragraph
  \renewcommand{\paragraph}{
    \@ifstar
      \xxxParagraphStar
      \xxxParagraphNoStar
  }
  \newcommand{\xxxParagraphStar}[1]{\oldparagraph*{#1}\mbox{}}
  \newcommand{\xxxParagraphNoStar}[1]{\oldparagraph{#1}\mbox{}}
\fi
\ifx\subparagraph\undefined\else
  \let\oldsubparagraph\subparagraph
  \renewcommand{\subparagraph}{
    \@ifstar
      \xxxSubParagraphStar
      \xxxSubParagraphNoStar
  }
  \newcommand{\xxxSubParagraphStar}[1]{\oldsubparagraph*{#1}\mbox{}}
  \newcommand{\xxxSubParagraphNoStar}[1]{\oldsubparagraph{#1}\mbox{}}
\fi
\makeatother

\usepackage{color}
\usepackage{fancyvrb}
\newcommand{\VerbBar}{|}
\newcommand{\VERB}{\Verb[commandchars=\\\{\}]}
\DefineVerbatimEnvironment{Highlighting}{Verbatim}{commandchars=\\\{\}}
% Add ',fontsize=\small' for more characters per line
\usepackage{framed}
\definecolor{shadecolor}{RGB}{241,243,245}
\newenvironment{Shaded}{\begin{snugshade}}{\end{snugshade}}
\newcommand{\AlertTok}[1]{\textcolor[rgb]{0.68,0.00,0.00}{#1}}
\newcommand{\AnnotationTok}[1]{\textcolor[rgb]{0.37,0.37,0.37}{#1}}
\newcommand{\AttributeTok}[1]{\textcolor[rgb]{0.40,0.45,0.13}{#1}}
\newcommand{\BaseNTok}[1]{\textcolor[rgb]{0.68,0.00,0.00}{#1}}
\newcommand{\BuiltInTok}[1]{\textcolor[rgb]{0.00,0.23,0.31}{#1}}
\newcommand{\CharTok}[1]{\textcolor[rgb]{0.13,0.47,0.30}{#1}}
\newcommand{\CommentTok}[1]{\textcolor[rgb]{0.37,0.37,0.37}{#1}}
\newcommand{\CommentVarTok}[1]{\textcolor[rgb]{0.37,0.37,0.37}{\textit{#1}}}
\newcommand{\ConstantTok}[1]{\textcolor[rgb]{0.56,0.35,0.01}{#1}}
\newcommand{\ControlFlowTok}[1]{\textcolor[rgb]{0.00,0.23,0.31}{\textbf{#1}}}
\newcommand{\DataTypeTok}[1]{\textcolor[rgb]{0.68,0.00,0.00}{#1}}
\newcommand{\DecValTok}[1]{\textcolor[rgb]{0.68,0.00,0.00}{#1}}
\newcommand{\DocumentationTok}[1]{\textcolor[rgb]{0.37,0.37,0.37}{\textit{#1}}}
\newcommand{\ErrorTok}[1]{\textcolor[rgb]{0.68,0.00,0.00}{#1}}
\newcommand{\ExtensionTok}[1]{\textcolor[rgb]{0.00,0.23,0.31}{#1}}
\newcommand{\FloatTok}[1]{\textcolor[rgb]{0.68,0.00,0.00}{#1}}
\newcommand{\FunctionTok}[1]{\textcolor[rgb]{0.28,0.35,0.67}{#1}}
\newcommand{\ImportTok}[1]{\textcolor[rgb]{0.00,0.46,0.62}{#1}}
\newcommand{\InformationTok}[1]{\textcolor[rgb]{0.37,0.37,0.37}{#1}}
\newcommand{\KeywordTok}[1]{\textcolor[rgb]{0.00,0.23,0.31}{\textbf{#1}}}
\newcommand{\NormalTok}[1]{\textcolor[rgb]{0.00,0.23,0.31}{#1}}
\newcommand{\OperatorTok}[1]{\textcolor[rgb]{0.37,0.37,0.37}{#1}}
\newcommand{\OtherTok}[1]{\textcolor[rgb]{0.00,0.23,0.31}{#1}}
\newcommand{\PreprocessorTok}[1]{\textcolor[rgb]{0.68,0.00,0.00}{#1}}
\newcommand{\RegionMarkerTok}[1]{\textcolor[rgb]{0.00,0.23,0.31}{#1}}
\newcommand{\SpecialCharTok}[1]{\textcolor[rgb]{0.37,0.37,0.37}{#1}}
\newcommand{\SpecialStringTok}[1]{\textcolor[rgb]{0.13,0.47,0.30}{#1}}
\newcommand{\StringTok}[1]{\textcolor[rgb]{0.13,0.47,0.30}{#1}}
\newcommand{\VariableTok}[1]{\textcolor[rgb]{0.07,0.07,0.07}{#1}}
\newcommand{\VerbatimStringTok}[1]{\textcolor[rgb]{0.13,0.47,0.30}{#1}}
\newcommand{\WarningTok}[1]{\textcolor[rgb]{0.37,0.37,0.37}{\textit{#1}}}

\usepackage{longtable,booktabs,array}
\usepackage{calc} % for calculating minipage widths
% Correct order of tables after \paragraph or \subparagraph
\usepackage{etoolbox}
\makeatletter
\patchcmd\longtable{\par}{\if@noskipsec\mbox{}\fi\par}{}{}
\makeatother
% Allow footnotes in longtable head/foot
\IfFileExists{footnotehyper.sty}{\usepackage{footnotehyper}}{\usepackage{footnote}}
\makesavenoteenv{longtable}
\usepackage{graphicx}
\makeatletter
\newsavebox\pandoc@box
\newcommand*\pandocbounded[1]{% scales image to fit in text height/width
  \sbox\pandoc@box{#1}%
  \Gscale@div\@tempa{\textheight}{\dimexpr\ht\pandoc@box+\dp\pandoc@box\relax}%
  \Gscale@div\@tempb{\linewidth}{\wd\pandoc@box}%
  \ifdim\@tempb\p@<\@tempa\p@\let\@tempa\@tempb\fi% select the smaller of both
  \ifdim\@tempa\p@<\p@\scalebox{\@tempa}{\usebox\pandoc@box}%
  \else\usebox{\pandoc@box}%
  \fi%
}
% Set default figure placement to htbp
\def\fps@figure{htbp}
\makeatother


% definitions for citeproc citations
\NewDocumentCommand\citeproctext{}{}
\NewDocumentCommand\citeproc{mm}{%
  \begingroup\def\citeproctext{#2}\cite{#1}\endgroup}
\makeatletter
 % allow citations to break across lines
 \let\@cite@ofmt\@firstofone
 % avoid brackets around text for \cite:
 \def\@biblabel#1{}
 \def\@cite#1#2{{#1\if@tempswa , #2\fi}}
\makeatother
\newlength{\cslhangindent}
\setlength{\cslhangindent}{1.5em}
\newlength{\csllabelwidth}
\setlength{\csllabelwidth}{3em}
\newenvironment{CSLReferences}[2] % #1 hanging-indent, #2 entry-spacing
 {\begin{list}{}{%
  \setlength{\itemindent}{0pt}
  \setlength{\leftmargin}{0pt}
  \setlength{\parsep}{0pt}
  % turn on hanging indent if param 1 is 1
  \ifodd #1
   \setlength{\leftmargin}{\cslhangindent}
   \setlength{\itemindent}{-1\cslhangindent}
  \fi
  % set entry spacing
  \setlength{\itemsep}{#2\baselineskip}}}
 {\end{list}}
\usepackage{calc}
\newcommand{\CSLBlock}[1]{\hfill\break\parbox[t]{\linewidth}{\strut\ignorespaces#1\strut}}
\newcommand{\CSLLeftMargin}[1]{\parbox[t]{\csllabelwidth}{\strut#1\strut}}
\newcommand{\CSLRightInline}[1]{\parbox[t]{\linewidth - \csllabelwidth}{\strut#1\strut}}
\newcommand{\CSLIndent}[1]{\hspace{\cslhangindent}#1}



\setlength{\emergencystretch}{3em} % prevent overfull lines

\providecommand{\tightlist}{%
  \setlength{\itemsep}{0pt}\setlength{\parskip}{0pt}}



 


\usepackage{graphicx}
\usepackage{eso-pic}
\usepackage{lastpage}
\usepackage{wallpaper} % for the background image on title page
\usepackage{fontawesome5}
\usepackage{xcolor}
\definecolor{orcidlogocol}{rgb}{0.65, 0.807, 0.223}
\definecolor{anugoldtint}{HTML}{F5EDDE}
\definecolor{anugold}{HTML}{BE830E}
\newcommand{\orcid}[1]{\href{https://orcid.org/#1}{\textcolor{orcidlogocol}{\faOrcid} #1}}

\usepackage[manualmark]{scrlayer-scrpage}
\clearpairofpagestyles
\ihead{}
\ofoot{Page \thepage\ of \pageref*{LastPage}}
\ifoot{The Australian National University\\{\footnotesize TEQSA Provider ID: PRV12002 (Australian University) | CRICOS Provider Code: 00120C}}
\addtokomafont{pageheadfoot}{\upshape}

\usepackage{multicol}

\usepackage[most]{tcolorbox}
\newtcolorbox{titlepagebox}{colback=anugoldtint,colframe=white}
\newtcbox{\inlinebox}[1][]{enhanced,
 box align=base,
 nobeforeafter,
 colback=white,
 colframe=gray,
 size=small,
 left=0pt,
 right=0pt,
 boxsep=2pt,
 #1}

% \usepackage{fancyhdr}
% % for title page
% \fancypagestyle{plain}{%
%   \renewcommand{\headrulewidth}{0pt}%
%   \fancyhf{}%
%   \fancyhf[rf]{Page \thepage\ of \pageref*{LastPage}}
% \fancyhf[lf]{The Australian National University\\{\footnotesize TEQSA Provider ID: PRV12002 (Australian University) | CRICOS Provider Code: 00120C}}
% }
% % for other pages
% \pagestyle{fancy}
% \fancyhf{}
% \fancyhf[rf]{Page \thepage\ of \pageref*{LastPage}}
% \fancyhf[lf]{The Australian National University\\{\footnotesize TEQSA Provider ID: PRV12002 (Australian University) | CRICOS Provider Code: 00120C}}
% \renewcommand{\headrulewidth}{0pt} % removes horizontal header line

%\usepackage{titling}
%\pretitle{% add some rules
%   \Huge\bfseries
%}%, make the fonts bigger, make the title (only) bold
%\posttitle{%
% \vskip .75em plus .25em minus .25em% increase the vertical spacing a bit, make this particular glue stretchier
%}

% ensure chapter starts on the same page
\makeatletter
\patchcmd{\scr@startchapter}{\if@openright\cleardoublepage\else\clearpage\fi}{}{}{}
\makeatother


\makeatletter
\@ifpackageloaded{caption}{}{\usepackage{caption}}
\AtBeginDocument{%
\ifdefined\contentsname
  \renewcommand*\contentsname{Table of contents}
\else
  \newcommand\contentsname{Table of contents}
\fi
\ifdefined\listfigurename
  \renewcommand*\listfigurename{List of Figures}
\else
  \newcommand\listfigurename{List of Figures}
\fi
\ifdefined\listtablename
  \renewcommand*\listtablename{List of Tables}
\else
  \newcommand\listtablename{List of Tables}
\fi
\ifdefined\figurename
  \renewcommand*\figurename{Figure}
\else
  \newcommand\figurename{Figure}
\fi
\ifdefined\tablename
  \renewcommand*\tablename{Table}
\else
  \newcommand\tablename{Table}
\fi
}
\@ifpackageloaded{float}{}{\usepackage{float}}
\floatstyle{ruled}
\@ifundefined{c@chapter}{\newfloat{codelisting}{h}{lop}}{\newfloat{codelisting}{h}{lop}[chapter]}
\floatname{codelisting}{Listing}
\newcommand*\listoflistings{\listof{codelisting}{List of Listings}}
\makeatother
\makeatletter
\makeatother
\makeatletter
\@ifpackageloaded{caption}{}{\usepackage{caption}}
\@ifpackageloaded{subcaption}{}{\usepackage{subcaption}}
\makeatother
\usepackage{bookmark}
\IfFileExists{xurl.sty}{\usepackage{xurl}}{} % add URL line breaks if available
\urlstyle{same}
\hypersetup{
  pdftitle={ANU Report Template},
  pdfauthor={Bill Gates; Prof.~Florence Nightingale; Ronald Fisher; Chat GPT},
  pdfkeywords={reproducible research, open science},
  hidelinks,
  pdfcreator={LaTeX via pandoc}}


\usepackage{tikz}
\AddToHook{shipout/foreground}{
            \begin{tikzpicture}[overlay, remember picture]
            \node[opacity=0.3, text=red, rotate=45, scale=6] at (current page) {DRAFT};
            \end{tikzpicture}
}

\title{ANU Report Template}
\usepackage{etoolbox}
\makeatletter
\providecommand{\subtitle}[1]{% add subtitle to \maketitle
  \apptocmd{\@title}{\par {\large #1 \par}}{}{}
}
\makeatother
\subtitle{via Quarto}
\author{Bill Gates \and Prof.~Florence Nightingale \and Ronald
Fisher \and Chat GPT}
\date{2nd January 2024}
\begin{document}
% Inspired by https://github.com/nmfs-opensci/quarto_titlepages_v1
  \begin{frontmatter}
  \begin{titlepage}
  % This is a combination of Pandoc templating and LaTeX
  % Pandoc templating https://pandoc.org/MANUAL.html#templates
  % See the README for help

  \newgeometry{top=3in,bottom=1in,right=1in,left=1in}
  \begin{minipage}[b][\textheight][s]{\textwidth}
  \raggedright

  % background image
  \ThisULCornerWallPaper{1}{\_extensions/anu-report/assets/images/coverpage.jpg}

  % Title and subtitle
  {\Huge\bfseries{ANU Report Template}}\\[1\baselineskip]
  {\LARGE{via Quarto}}\\[1\baselineskip]
      \inlinebox{\scriptsize\textcolor{gray}{\MakeUppercase{optional
  category labels}}}
      \inlinebox{\scriptsize\textcolor{gray}{\MakeUppercase{technical
  report series}}}
    \vspace{3mm}

      This report template is available at
      \href{https://github.com/anuopensci/quarto-anu-report}{github.com/anuopensci/quarto-anu-report}.

  \begin{titlepagebox}



  % Authors
  \textcolor{anugold}{\MakeUppercase{Authors}}
  \vspace{-3mm}
  \begin{multicols}{2}
      \begin{minipage}{\columnwidth}
      \raggedright
      \normalfont
      {\textbf{Bill Gates}}\\
          
      {\itshape{Bill \& Melinda Gates Foundation}}\\
          \faIcon[regular]{envelope} bill@gates.com\\
      
      \orcid{0000-0003-1689-0557}\\
      \vspace{4mm}
      \end{minipage}

      \begin{minipage}{\columnwidth}
      \raggedright
      \normalfont
      {\textbf{Prof.~Florence Nightingale}}\\
          {Research School of Visualiation}\\
      {\itshape{Australian National University}}\\
          \faIcon[regular]{envelope} florence@anu.edu.au\\
      
      
      \vspace{4mm}
      \end{minipage}

      \begin{minipage}{\columnwidth}
      \raggedright
      \normalfont
      {\textbf{Ronald Fisher}}\\
          {Frequentist Statistics Institute}\\
      {\itshape{Australian National University}}\\
          \faIcon[regular]{envelope} ronald@anu.edu.au\\
      
      
      \vspace{4mm}
      \end{minipage}

      \begin{minipage}{\columnwidth}
      \raggedright
      \normalfont
      {\textbf{Chat GPT}}\\
          
      {\itshape{OpenAI}}\\
          
      
      
      \vspace{4mm}
      \end{minipage}

  \end{multicols}

  \vspace{-4mm}

  % Date %

  % FIXME: \bfseries doesn't seem to be working
  {\textcolor{anugold}{\MakeUppercase{Date}}}\\ 2nd January 2024
  {\itshape{(Last modified: 23rd June 2025)}}
  \vspace{3mm}

  \textcolor{anugold}{\MakeUppercase{Executive Summary}}\\
  The report delineates key principles, such as version control and data
  sharing, and introduces practical tools like RMarkdown, Jupyter
  Notebooks, and Docker. Reproducible research significantly boosts the
  credibility of scientific findings, fostering collaboration, iterative
  improvements, and efficient communication. The document addresses
  challenges like data privacy and infrastructure constraints, proposing
  solutions such as anonymization and institutional support. Recognizing
  the transformative impact of reproducibility, the report concludes
  that its adoption contributes to building a culture of transparency
  and accountability in scientific endeavors, laying the groundwork for
  a more robust foundation in advancing knowledge. Recommendations
  include providing training resources, support infrastructure, and
  guidelines to facilitate the integration of reproducible research
  practices into standard workflows.
  \vspace{3mm}

  {\textcolor{anugold}{\MakeUppercase{Keywords}}}\\
  reproducible research, open science

  \end{titlepagebox}

  \vspace{2\baselineskip}





  %use \vfill instead to get the space to fill flexibly
  %\vspace{0.25\textheight} % Whitespace between the title block and the publisher

  \vfill

  %%%%%% Cover image


  % Whitespace between the title block and the tagline
  \vspace{1\baselineskip}

  The Australian National University\\{\footnotesize TEQSA Provider ID: PRV12002 (Australian University) | CRICOS Provider Code: 00120C}

  \end{minipage}
  \restoregeometry

  \end{titlepage}
  \end{frontmatter}

\restoregeometry

\cleardoublepage
\AddToShipoutPicture{\includegraphics[width=\paperwidth,height=\paperheight]{\_extensions/anu-report/assets/images/page-background.jpg}}


\floatplacement{table}{H}
\renewcommand*\contentsname{Table of contents}
{
\setcounter{tocdepth}{2}
\tableofcontents
}

\mainmatter
\section{Introduction}\label{introduction}

Reproducible research is an essential paradigm that promotes the idea
that scientific investigations should be transparent, verifiable, and
accessible to others. In an era where the scientific community faces
concerns about the replicability of research findings, adopting
reproducible research practices becomes imperative.

This document uses Allaire (2023).

\section{Key Benefits of Reproducible
Research}\label{key-benefits-of-reproducible-research}

\begin{itemize}
\tightlist
\item
  \emph{Increased Credibility}: Transparent and reproducible research
  enhances the credibility of scientific findings, as others can
  independently verify and validate the results.
\item
  \emph{Collaboration and Iteration}: Open sharing of code and data
  facilitates collaboration among researchers and allows for the
  iterative improvement of studies over time.
\item
  \emph{Efficient Communication}: Reproducible research documents serve
  as comprehensive and self-contained artifacts, aiding in effective
  communication of methodologies and results.
\end{itemize}

\section{Method}\label{method}

\subsection{Getting started}\label{getting-started}

\begin{enumerate}
\def\labelenumi{\arabic{enumi}.}
\tightlist
\item
  Install \href{https://quarto.org/docs/get-started/}{Quarto}.
\item
  You may need to install Public Sans font if you do not have it. You
  can find it
  \href{https://github.com/anuopensci/quarto-anu-report/tree/main/_extensions/anu-report/assets/webfonts}{here}.
\item
  The easiest way to edit and render the document is using the
  \href{https://posit.co/download/rstudio-desktop/}{RStudio IDE}. If you
  prefer not to install R, then you can use VS Code instead.
\item
  From the Terminal run the following command
\end{enumerate}

\begin{Shaded}
\begin{Highlighting}[]
\ExtensionTok{quarto}\NormalTok{ use template anuopensci/quarto{-}anu{-}report}
\end{Highlighting}
\end{Shaded}

\section{Conclusion}\label{conclusion}

This report underscores the transformative impact of reproducible
research on the scientific landscape, promoting a commitment to openness
and accountability for the benefit of the entire research community.

\section*{References}\label{references}
\addcontentsline{toc}{section}{References}

\phantomsection\label{refs}
\begin{CSLReferences}{1}{0}
\bibitem[\citeproctext]{ref-quarto}
Allaire, JJ. 2023. \emph{Quarto: R Interface to 'Quarto' Markdown
Publishing System}. \url{https://CRAN.R-project.org/package=quarto}.

\end{CSLReferences}


\backmatter


\end{document}
